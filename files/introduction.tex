\section{Purpose}
In this section, I want to introduce briefly my application.\\

The application is developed for the course of \textit{Design and Implementation of Mobile
Applications} of professor Luciano Baresi at Politecnico di Milano.\\
The goal of the course is to design and implement a mobile application on a platform of our choice.
This document illustrates the decisions I made in order to accomplish this goal.\\

This Software Design Document is a document that provides documentation that will be
used as overall guidance to the architecture of the software project. Moreover, I will
provide documentation of the software design of the project, including use case models,
class and sequence diagrams.\\

The purpose of this document is to provide a full description of the design of \textit{MOQA}, a cross-platform application, providing insights into the structure and design of each component.

\section{Scope}
The application comes by a need of professor Fulvio Re Cecconi (ABC Departement of Polimi).\\
He made an Arduino board that allows measuring some
air quality parameters like temperature, relative humidity, light, etc... .Measures are GPS located and saved to a microSD card.\\

The aim is to increase the accuracy of the GPS location using GNSS Real Time Kinematics (RTK); the increased precision is obtained thanks to a GPS correction signal transmitted via the internet. Thus Arduino must be equipped with a connection and use a mobile phone as a modem. Moreover, an application on a phone have to produce dynamic maps and charts.\\

During the initial meetings, we also decided to include a dynamic graph and compare Arduino data with the data registered by ARPA.

\section{Stakeholders}
The main stakeholders of my project are Professor Re Cecconi, students and people that have to take measures with the Arduino board.\\

Moreover, I have to consider, as a stakeholder, professor Luciano Baresi that holds the \textit{Design and Implementation of Mobile Applications} course and Giovanni Quattrocchi, teaching assistant of the course.

\section{Time Constraints}
I had no precise and punctuated deadlines, but to deliver the project among the different call dates for the course.\\

Professor Re Cecconi asks for an application as soon as possible; however, without fixing a precise deadlines.\\

Initial meeting was done at the end of November 2019 and a fist mock-up was provided in the mid of December 2019. During Christmas vacations and in the free-time in January and February I spent time learning the different frameworks used.\\
I decided to plan the start of development at the beginning of the second semester. Developing, testing and creating documentation is planned to last less than 6 months. Even the server-side part will build and maintained during the same time.\\

I started development in February 2020 and complete it in June 2020.

\section{Risk Analysis}
In the first phase of analysis, I try to exploit the possible risks that can compromise the project development.\\

Risk Analysis was made in collaboration with the other two students that decided to develop the same project.\\

We had to be very careful during the first phases of implementation, especially for the Arduino code, to manage all possible issues, leading to possible code rewriting and the inevitable loss of time.

\section{Overview}
This document is structured as follows:
\begin{itemize}
  \setlength{\itemindent}{-.4in}
  \item[] \textbf{Section 1: Introduction}. A general introduction of the Design and Technology Document. It aims giving general but exhaustive information about what this document is going explain.
  
  \item[] \textbf{Section 2: General Overview}. A general overview of the project. In this section, the reader could find the core features of the application and the requirements of the system.
  
  \item[] \textbf{Section 3: Architectural Design}. This section contains an overview of the high-level components of the system and then a more detailed description of three architecture views: Component view, Deployment View and Runtime View. Finally, it shows the Component Interfaces and the chosen architecture styles and patterns.
  
  \item[] \textbf{Section 4: User Interface Design}. This section contains the screenshots of the application with some comments to give to the reader a general overview of the user interfaces.
  
  \item[] \textbf{Section 5: Frameworks, External services and Libraries}. This section aims to explain the main frameworks, external services and libraries used pointing out their advantages.
  
  \item[] \textbf{Section 6: Test Case}. This section identifies test cases performed reporting their results.
  
  \item[] \textbf{Section 7: Effort and Cost Estimation}. A summary of worked time and cost estimation of the work.
  
  \item[] \textbf{Section 8: Future Works}. This section contains some new features that, in future, could be add to \textit{MOQA} or the \textit{Application Server}.
\end{itemize}

At the end, there is an \textbf{Appendix} where software and tools used are reported.
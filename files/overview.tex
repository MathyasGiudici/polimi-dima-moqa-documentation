\section{Concept}
\textit{MOQA} is a cross-platform application that allows to retrieve air parameters registered by an Arduino board, send them to a remote server and visualize through a map and a chart.\\

The application requires users to be authenticated. This strong requirement because to retrieve data and to send data to the remote server you need an authentication. \\ 
Without these data the application could not provide any type of feature it has designed to; so it has no sense maintain a part without authentication.\\

The main components of the project are:
\begin{itemize}
    \item A client-side part, that is \textit{MOQUA}, which queries the database or retrieves data from Arduino, visualizes data on a map and a chart, send new Arduino data to the server;
    \item A server-side part, composed by a relational database that contains air quality data and users data; moreover it contains the application logic that delivers the RestfulAPI.
\end{itemize}

\begin{figure}[h]
\begin{center}
  \includegraphics{img/logo_moqa.png}
  \hspace{0.05\linewidth}
  \centering
  \caption{\textit{MOQA} logo}
  \label{img:logo_moqa}
\end{center}
\end{figure}